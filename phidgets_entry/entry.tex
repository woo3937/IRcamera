% a contest found at http://www.phidgets.com/phorum/viewtopic.php?f=52&t=6299
% grand prize: 1000 worth of phidgets store credit
% What is expected for submissions:
    %A video demonstrating your project (ideally a link to a YouTube or Vimeo video)
    %A brief description of the project
    %Pictures
    %Code, schematics or other documents

\documentclass{article}
\usepackage{hyperref}
\usepackage[usenames,dvipsnames]{xcolor}
\hypersetup{
    unicode=false,                % non-Latin characters in Acrobat’s bookmarks
    pdftoolbar=true,              % show Acrobat’s toolbar?
    pdfmenubar=true,              % show Acrobat’s menu?
    pdffitwindow=false,           % window fit to page when opened
    pdfstartview={FitH},          % fits the width of the page to the window
    pdftitle={CV},                % title
    pdfauthor={Scott Sievert},    % author
    pdfsubject={},                % subject of the document
    pdfcreator={Scott Sievert},   % creator of the document
    pdfproducer={Scott Sievert},  % producer of the document
    pdfkeywords={compressed sensing,} {research,} {academia,}, % list of keywords
    pdfnewwindow=true,            % links in new window
    colorlinks=true,              % false: boxed links; true: colored links
    linkcolor=red,                % color of internal links
    citecolor=green,              % color of links to bibliography
    filecolor=magenta,            % color of file links
    urlcolor=NavyBlue             % color of external links
}

\newcommand{\github}{https://github.com/scottsievert/IRcamera/tree/master/temp.rpi/IRcamera}

\begin{document}

    Normally, infa-red (IR) cameras cost between 4,000 and 40,000 dollars. We're making a camera that uses a single sensor to take a picture -- a simple project anyone could do. But, on top of it, we're adding a compressed sensing algorithm to reduce the exposure time. Our final camera is around 400 dollars. Because we're using a compressed sensing algorithm, we have to use extremely precise motors, so we're using two Phidget stepper motors that give percision down to three tenths of a degree.

    All of the code can be found \href{\github}{on my GitHub.} Specifically, look in \texttt{brain.c}. Also, we're written good documentation; we hate it when other people don't write docs all the way.

    % include a photo
    % include a video of the camera shooting



\end{document}


